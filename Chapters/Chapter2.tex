\chapter{Background}
\label{chap:background}
\lhead{\emph{Background}}

\section{Thematic Area within Computer Science}
The core topic of this project is to develop a web application where users can manage their details across multiple servers. It will be an administrative application so the basic functionality won't be too complex. Where the complexity comes in is having a secure database and our application interacting with varies servers. 

%\section{Project Scope}
%Project specifics: Background minimum knowledge.

%Imagine you wanted to explain the specifics of your project to a person that knows nothing of Computer Science. You cannot talk about everything (as the idea is not to write a 500+ page report). Remember the reader at this stage can only be assumed to know what you have covered, so identify what are the minimum concepts belonging to the main areas (listed as 3 in the section before) and the core areas (listed as 2 in the section before) that you would need to explain so that the reader is able to understand the specifics of your project and indeed the following section. For example the minimum amount of knowledge about software development, cloud computing, mobile applications, social networking and service providers that are required so as to understand the specifics of a project about a mobile app for an on-line voting system. Here we are making the same trip we did before, but now in the opposite direction. Start zooming in from 3, then to 2 and finally to reach your project 1. Once the reader is finished this section they should be able to understand the proceeding sections (and have context for it within the project).

\section{A Review of User Management Systems}
Our application will be handling many users but we need to understand what these users are doing to know how to handle them. They are all part of a translation management system that globalizes and then localizes websites, apps etc. 

A translation management system is a type of software used for automating parts of the human language. The idea of a TMS is simple, it’s to automate repetitive and non-crucial work leaving just the creative and review aspects to be done manually. A typical TMS monitors the source language for any changes and if there are any changes translators and reviewers are notified. 

There are many benefits to using these types of systems but the main one is that a product is easily accessible on a global scale. The localization process being automated as much as possible drastically reduces management, overhead and time spent. Using this type of work flow brings accountability into the frame too. When a files source language is changed who and when it was changed is known. Then when source language needs to be translated, a job is created for this with a unique ID, a translator then is given this job and gives an expected completion date. 
