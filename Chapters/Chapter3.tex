\chapter{Problem - GlobalSight User Management System}
\label{chap:problem}
\lhead{\emph{Problem Statement}}

\section{Problem Definition}
GlobalSight is a Globalization Management System. It contains a translation management system, and a complimentary work-flow management system. It’s the primary tool for getting McAfee productions, documentation, and websites translated. McAfee translate into over 60 languages, and variants. GlobalSight is also used as a LDAP substitute for other internal systems, where integration with Active Directory is not possible, or desired.

Multiple servers are used to share load, and to ensure segregation of business content streams. McAfee also have separate development and staging/UAT environments. There are currently twelve separate servers in use. Production server, at least, of each content stream is externally facing, and is accessed by translators worldwide, to complete their translation work.

At present, more than 1500 unique active users on the system, between translators and internal users. Each user is on each of the 12+ servers. When a user is created, or their password
updated, that change needs to be replicated to each server.

It currently takes about 25 minutes to change 1 password, and about 30 minutes to create a single use on all systems.


\section{Objectives}
To create a web portal application that will allow easy management of user in the GlobalSight environment.  GlobalSight has a published API set, which includes API’s to manage users, which should be used by the application.

The application should have the ability to adjust dynamically as more servers are added to the cluster, and where some are removed/decommissioned. 

The application will be safe from a security point of view, having so many users data protection is key. Mcafees' own security team will be running regular checks for vulnerabilities in the system.

\section{Functional Requirements}
The application will have the ability to differentiate between a normal user and an administrators. Administrators should have the ability to which users can log in to the system and disable a user's log in if required. Administrators will also have the ability to manage the server list.

The application be able to define server groups, so users can group the production, development and UAT servers of a content stream together. A server can only exist in one group. This includes the main server, it’s details are not stored in the database, but it’s assignable to a group. A server must be part of a group, but a group can contain one server. There’s no limit to the number of servers in a group. A group can have zero members, and not show in the UI for app users, but still exist. It should be possible to delete or make inactive a group. Deleting a group removes it from the database, while machining active just removes sets an “inactive” flag in the database. There’s no need for a UI to show inactive servers, not is there a need for functionality to make them active again. That can be done via database directly. A group cannot be deleted or made inactive, if it still has members. 

\pagebreak


\section{Non-Functional Requirements}
The UI for will be minimalistic, allowing for users to easily traverse the application in the most efficient way.  
All content such as strings, images and colours will be externalized to separate files to allow for the application to be fully translated in future. 

The application will be reliable, users cannot access their accounts and the application is down for some reason that could potentially leave workers with no ability to actually achieve their work.




